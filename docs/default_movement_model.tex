\documentclass[10pt,letterpaper]{article}

% Packages
\usepackage[utf8]{inputenc}
\usepackage[top=0.85in,left=1.5in,footskip=0.75in]{geometry}
\usepackage{lastpage,fancyhdr,graphicx}
\usepackage{epstopdf}
\usepackage{verbatim}
\usepackage{amsmath}
\usepackage{amsfonts}
\usepackage{hyperref}
\usepackage{setspace}

% Text layout
\raggedright
\setlength{\parindent}{0.5cm}
\textwidth 5.5in 
\textheight 8.75in
\onehalfspacing

% Bibliography
\usepackage[backend=biber,style=numeric,sorting=ynt]{biblatex}
\addbibresource{bibliography.bib}

% C++ symbol
\newcommand{\CC}{C\nolinebreak\hspace{-.05em}\raisebox{.4ex}{\tiny\bf +}\nolinebreak\hspace{-.10em}\raisebox{.4ex}{\tiny\bf +}}
\def\CC{{C\nolinebreak[4]\hspace{-.05em}\raisebox{.4ex}{\tiny\bf ++}}}

% Header and footer
\pagestyle{fancy}
\fancyhf{}
\rfoot{\thepage/\pageref{LastPage}}
\renewcommand{\headrulewidth}{0pt}
\fancyheadoffset[L]{2.25in}
\fancyfootoffset[L]{2.25in}

\begin{document}

\vspace*{0.2in}

\begin{centering}
{\huge\textbf\newline{Pandemia: Default Movement Model}}
\\
\bigskip
\includegraphics[width=0.2\textwidth]{pandemia_logo}
\\
\bigskip
\today
\\
\end{centering}

\subsection*{Introduction}

The Pandemia model acts on a \textbf{World}, with each \textbf{World} consisting of \textbf{Regions}, with each \textbf{Region} consisting of \textbf{Individuals}, \textbf{Locations} and \textbf{Activities}. Each individual performs a sequence of activities and performs these activities at particular locations. During the building of the \textbf{World}, each individual is assigned a weekly routine, starting on a Sunday, consisting of a sequence of activities.

For example, suppose in a given region there are three activities, \textbf{Home}, \textbf{Work} and \textbf{School}, and that tick length is set equal to 8 hours, meaning that there are 3 ticks of the simulation clock per day. Then a possible weekly routine could be:
\begin{center}
[\textbf{Home}, \textbf{Home}, \textbf{Home}, \textbf{Home}, \textbf{Work}, \textbf{Home}, \textbf{Home}, \textbf{Work}, \textbf{Home}, \textbf{Home}, \textbf{Work}, \textbf{Home}, \textbf{Home}, \textbf{Work}, \textbf{Home}, \textbf{Home}, \textbf{Work}, \textbf{Home}, \textbf{Home}, \textbf{Home}, \textbf{Home}]
\end{center}
representing a typical working week, or alternatively 
\begin{center}
[\textbf{Home}, \textbf{Home}, \textbf{Home}, \textbf{Home}, \textbf{School}, \textbf{Home}, \textbf{Home}, \textbf{School}, \textbf{Home}, \textbf{Home}, \textbf{School}, \textbf{Home}, \textbf{Home}, \textbf{School}, \textbf{Home}, \textbf{Home}, \textbf{School}, \textbf{Home}, \textbf{Home}, \textbf{Home}, \textbf{Home}]
\end{center}
representing a typical school week.

During the building of the \textbf{World}, each individual is assigned a weighted set of locations for each activity. These are the locations at which an individual can perform each activity. During the simulation, whenever an individual switches from one activity to another, the individual randomly selects a new location from the set of allowed locations, according to the weights.

For example, suppose in a given region there are two activities, \textbf{Home} and \textbf{Other}. Then, a given individual might be assigned a single location for the activity \textbf{Home}, with weight $1.0$, and several other locations for the activity \textbf{Other}, with weights chosen according to the distance of each location from the home, so that locations further from home are less likely to be visited, and so on.

Upon changing activity, individuals may also put on or take off a face mask, depending on the activity and the current policy on face masks.

If the change of location is prohibited by policy interventions, then the individual is directed to their home location. In particular, each individual must for the default movement model be assigned a home location.
\end{document}